\documentclass[a4paper,12pt]{article}
\usepackage[utf8]{inputenc}
\usepackage[spanish]{babel}
\usepackage{graphicx}
\usepackage[margin=3cm]{geometry}
\usepackage{hyperref}
\usepackage{parskip}

\hypersetup{%
    pdfborder = {0 0 0}
}

\date{}

\title{\textbf{La paradoja de Aquiles y la Tortuga}}

\author{\begin{tabular}{rl}
  \textbf{Autor:} & Alba Puelles López \\
  \textbf{Correo:} & alba.puelles@um.es \\
  \textbf{Grupo:} & 1.1 \\
  \textbf{Profesor:} & Alberto Ruiz García \\
\end{tabular}}

\begin{document}
 
   \maketitle
   
   \newpage

   \tableofcontents

   \newpage
   
   \section{Introducción}
   
   En este documento presentaremos una paradoja que intenta contradecir
   el siguiente hecho: un corredor veloz alcanzará a uno lento aunque le dé ventaja. 
   Definiremos y desmostraremos la Paradoja de Aquiles y su contraparadoja, que 
   dice que el corredor veloz consigue alcanzar en un punto al lento, pero que aún así,
   no consigue ganarle.
   
   Esto será demostrado desde un punto de vista matemático porque como bien sabemos, 
   mirándolo desde la física (un corredor muy rápido puede adelantar
   a uno más lento aunque le de ventaja) o hipotetizando algunas situaciones que pudieran pasar
   como que el corredor lento se quedara sin aliento y abandonara la carrera, 
   el rápido sí que podría ganar la carrera.
   
   \section{Definición del problema}
   
   Aquiles, conocido como ``el de los pies ligeros" debido a que se le consideraba 
   el hombre más veloz y el más hábil guerrero, decide participar en una carrera 
   contra una tortuga. Debido a que Aquiles es mucho más rápido que la tortuga, 
   decide darle una gran ventaja inicial. Al comienzo de la carrera, 
   Aquiles recorre en poco tiempo la distancia de ventaja que le había dejado a la tortuga, 
   pero al llegar descubre que la tortuga ya no está, sino que ha avanzado, más lentamente,
   un poco más. Muy decidido y con ánimo, sigue corriendo, pero al llegar de nuevo donde 
   estaba la tortuga la última vez que miró, ésta ha avanzado un poco más. 
   De esta manera, Aquiles nunca consigue ganar la carrera.
   
   \section{Demostración}
   
   \section{Demostración de la falsedad de la hipótesis}
      



   
   \newpage
   
   \begin{thebibliography}{99}
   
      \bibitem{Wikipedia} https://es.wikipedia.org/wiki/Paradojas\_de\_Zenon
      
      
   \end{thebibliography}
  
\end{document} 
