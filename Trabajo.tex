\documentclass[a4paper,12pt]{article}
\usepackage[utf8]{inputenc}
\usepackage[spanish]{babel}
\usepackage{graphicx}
\usepackage[margin=3cm]{geometry}
\usepackage{hyperref}
\usepackage{parskip}

\hypersetup{%
    pdfborder = {0 0 0}
}

\date{}

\title{\textbf{La paradoja de Aquiles y la Tortuga}}

\author{\begin{tabular}{rl}
  \textbf{Autor:} & Alba Puelles López \\
  \textbf{Correo:} & alba.puelles@um.es \\
  \textbf{Grupo:} & 1.1 \\
  \textbf{Profesor:} & Alberto Ruiz García \\
\end{tabular}}

\begin{document}
 
   \maketitle
   
   \newpage

   \tableofcontents

   \newpage
   
   \section{Introducción}
   
   Aquiles, conocido como ``el de los pies ligeros" debido a que se le consideraba 
   el hombre más veloz y el más hábil guerrero, decide participar en una carrera 
   contra una tortuga. Debido a que Aquiles es mucho más rápido que la tortuga, 
   decide darle una gran ventaja inicial. Al comienzo de la carrera, 
   Aquiles recorre en poco tiempo la distancia de ventaja que le había dejado a la tortuga
   ,pero al llegar descubre que la tortuga ya no está, sino que ha avanzado, más lentamente,
   un poco más. Muy decidido y con ánimo, sigue corriendo, pero al llegar de nuevo donde estaba la tortuga
   la última vez que miró, ésta ha avanzado un poco más.
   
   En este trabajo demostraremos que Aquiles no ganará la carrera y que 
   la tortuga estará siempre por delante de él. Esto será demostrado desde un punto de vista matemático
   porque como bien sabemos, mirándolo desde la física (un corredor muy rápido puede adelantar
   a uno más lento aunque le de ventaja) o hipotetizando algunas situaciones que pudieran pasar
   como que la tortuga se quedara sin aliento y abandonara la carrera, Aquiles sí que podría ganar la carrera.
   
   \section{Definición del problema}
   
   
   
   \section{Demostración}
   
   \section{Demostración de la falsedad de la hipótesis}
      



   
   \newpage
   
   \begin{thebibliography}{99}
   
      \bibitem{Wikipedia} https://es.wikipedia.org/wiki/Paradojas\_de\_Zenon
      
      \bibitem{paradoja} https://chemazdamundi.wordpress.com/2010/09/07/
      falacias-economicas-iii-la-importancia-del-lenguaje-matematico-en-la
      -formulacion-cientifica-economica-la-paradoja-de-auiles-y-la-tortuga-
      y-la-escuela-de-austria-contrastada-pseudociencia-en-econ/
      
   \end{thebibliography}
  
\end{document} 
